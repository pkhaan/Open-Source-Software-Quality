%%%%%%%%%%%%%%%%%%%%%%%%%%%%%%%%%%%%%%%%%%%%%%%%%%%%%
%% File: main.tex
%% Authors: Petros Chanas, Konstantinos Vasilopoulos
%% Last update: April, 2022
%% Description: Provides our assignment 
%% regarding Open Source Software Quality Testing
%%and parameters of rating the performance and quality
%%of software developed of open source applications
%%%%%%%%%%%%%%%%%%%AUEB2021-2022%%%%%%%%%%%%%%%%%%%%%
%% Character encoding: UTF-8
%%%%%%%%%%%%%%%%%%%%%%%%%%%%%%%%%%%%%%%%%%%%%%%%%%%%%


\documentclass{article}
% Set the font (output) encoding
\usepackage[LGR]{fontenc}

% Greek-specific commands
\usepackage[greek]{babel}
\usepackage{graphicx}
\graphicspath{ {./images/} }
\title{\textbf{\textlatin{Open source software quality}}}
\author{{\textit{Κωνσταντίνος Βασιλόπουλος\(\bullet\)3180018}\\*\textit{Πέτρος Χάνας\(\bullet\)3170173}}\\*\includegraphics[width=4cm, height=3cm]{oSource}}
\date{\today}
\begin{document}
\maketitle

\begin{abstract}
Σε αυτήν την εργασία θα αναφερθούμε στους τρόπους ανάπυτηξης ελεύθερου λογισμικού
και στα ενδεχόμενα προβλήματα τα οποία προκύπτουν ως προς τον προγραμματισμό και το 
\textlatin{debugging} αυτού καθ'όλη τη διάρκεια ζωής του. Επίσης θα μελετήσουμε τους 
τρόπους αξιολόγησης του λογισμικού από τρίτους παράγοντες όπως εταιρείες, δημόσιοι οργανισμοί 
αλλά και από τους ίδιους τους χρήστες. Ουσιαστικα, η ποιότητα του ανοιχτού-ελεύθεροου λογισμικού 
συνίσταται από κάποιους βασικούς πυλώνες άμεσα εξαρτώμενος από το ίδιο το λογισμικό όπως η ταχύτητά
του, η αξιοπιστία του, η ευκολία στην χρήση του αλλά και την περιήγησή του και πολλά άλλα. Εν ολίγοις, 
θα αναλύσουμε όλες τις πτυχές ανάπτυξης και αξιολόγησης της ποιότητας λογισμικού ανοιχτού κώδικα 
διανθίζοντας όλα τα επιμέρους στοιχεία που συγκροτούν αλλά και καθιστούν το λογισμικό έτοιμο προς διάθεση 
στο ευρύ κοινό.

\end{abstract}

\section{Τι είναι το λογισμικό ανοιχτού κώδικα;}
%%perigrafh open source code by definition
%%famous examples etc.
 Συχνά θα ακούσουμε ανθρώπους της πληροφορικής να αναφέρονται σε συγκεκριμένα προγράμματα ως «ανοιχτού κώδικα» ή «ελεύθερου λογισμικού». Εάν ένα πρόγραμμα είναι ανοιχτού κώδικα, ο πηγαίος κώδικας του είναι ελεύθερος διαθέσιμος στους χρήστες του. Οι χρήστες του - και οποιοσδήποτε άλλος - έχουν τη δυνατότητα να λάβουν αυτόν τον πηγαίο κώδικα, να τον τροποποιήσουν και να διανείμουν τις δικές τους εκδόσεις του προγράμματος. Οι χρήστες έχουν επίσης τη δυνατότητα να διανέμουν όσα αντίγραφα του αρχικού προγράμματος θέλουν. Οποιοσδήποτε μπορεί να χρησιμοποιήσει το πρόγραμμα για οποιονδήποτε σκοπό. δεν υπάρχουν χρεώσεις αδειοδότησης ή άλλοι περιορισμοί στο λογισμικό. Για παράδειγμα, το \textlatin{Ubuntu Linux} είναι ένα λειτουργικό σύστημα ανοιχτού κώδικα. Μπορείτε να κατεβάσετε το \textlatin{Ubuntu}, να δημιουργήσετε όσα αντίγραφα θέλετε και να τα δώσετε στους φίλους σας. Μπορείτε να εγκαταστήσετε το Ubuntu σε απεριόριστο αριθμό υπολογιστών. Μπορείτε να δημιουργήσετε αντίγραφα του δίσκου εγκατάστασης του \textlatin{Ubuntu} και να τα διανείμετε. Εάν είχατε ιδιαίτερα κίνητρα, θα μπορούσατε να κατεβάσετε τον πηγαίο κώδικα για ένα πρόγραμμα στο Ubuntu και να το τροποποιήσετε, δημιουργώντας τη δική σας προσαρμοσμένη έκδοση αυτού του προγράμματος - ή του ίδιου του \textlatin{Ubuntu}. Όλες οι άδειες ανοιχτού κώδικα σάς επιτρέπουν να το κάνετε αυτό, ενώ οι άδειες κλειστού κώδικα επιβάλλουν περιορισμούς σε εσάς.
 



\section{Ιστορική Αναδρομή}

Κατά τη διάρκεια των πρώτων δεκαετιών της πληροφορικής, ήταν ο κανόνας και όχι η εξαίρεση η κοινή χρήση αναγνώσιμου πηγαίου κώδικα από τον άνθρωπο. Στις δεκαετίες του 1950 και του 1960 σχεδόν όλο το λογισμικό που παρήχθη από ακαδημαϊκούς και εταιρικά ερευνητικά εργαστήρια, όπως τα \textlatin{Bell Labs} της  \textlatin{AT&T}, εργάστηκαν σε συνεργασία. Όλοι είχαν μακροχρόνιες παραδόσεις ανοιχτού χαρακτήρα και συνεργασίας, άτυπες για τον ακαδημαϊκό χώρο, και ως εκ τούτου, ακόμη κι αν το λογισμικό δεν ήταν επίσημα διαθέσιμο στο κοινό, ο πηγαίος κώδικας τους ήταν ευρέως κοινός. Οι εταιρείες ηλεκτρονικών υπολογιστών διένειμαν επίσης τον πηγαίο κώδικα του λογισμικού που έστειλαν μαζί με το υλικό, για δύο λόγους. Πρώτον, δεν έβλεπαν το λογισμικό ως εμπόρευμα προς πώληση μέσω αδειοδότησης. Δεύτερον, οι χρήστες τροποποιούσαν συχνά οι ίδιοι το λογισμικό επειδή δεν θα λειτουργούσε σε διαφορετικό υλικό ή λειτουργικό σύστημα χωρίς κάποια παρέμβαση στον "κορμό" του, καθώς και για να διορθώσουν σφάλματα ή να προσθέσουν νέες λειτουργίες.

Το να έχουν διαθέσιμο τον πηγαίο κώδικα ήταν μια αναγκαιότητα για τους κατασκευαστές υπολογιστών, καθώς η δημιουργία διαφορετικών \textlatin{binary files} (ή μεταγλωττισμένου κώδικα) για διαφορετικό υλικό δεν ήταν καθόλου πρακτική και 
συχνά δημιουργούσε μεγάλες επιπλοκές. Ορισμένα πανεπιστήμια είχαν ακόμη και μια πολιτική που απαιτούσε όλα τα λογισμικά που είναι εγκατεστημένα στους υπολογιστές στα εργαστήριά τους να συνοδεύονται από δημοσιευμένους πηγαίους κώδικες.

Το 1953, το τμήμα \textlatin{UNIVAC} του \textlatin{Remington Rand} ανέπτυξε την πρώτη παρουσία ελεύθερου λογισμικού ανοιχτού κώδικα που ονομάζεται  \textlatin{A-2 (Arithmetic Language v2 system)}, το οποίο κυκλοφόρησε στους πελάτες τους μαζί με τον πηγαίο κώδικα. Προσκλήθηκαν επίσης να στείλουν πίσω τις βελτιώσεις τους. Αργότερα, το πρώτο λειτουργικό σύστημα της  \textlatin{IBM}, ο κώδικας του  \textlatin{IBM 704} διανεμήθηκε με όλους τους μεγάλους υπολογιστές τους. Οργανισμοί όπως η \textlatin{IBM}, η \textlatin{DEC} και η \textlatin{General Motors} δημιουργούν ομάδες χρηστών για να διευκολύνουν την κοινή χρήση κώδικα μεταξύ των χρηστών, ακαδημαϊκών και άλλων παραγόντων του κλάδου.







\subsection{Σύντομη Αναφορά στα \textlatin{Linux}}
%%linux milestone









\end{document}