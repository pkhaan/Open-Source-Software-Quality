%%%%%%%%%%%%%%%%%%%%%%%%%%%%%%%%%%%%%%%%%%%%%%%%%%%%%
%% File: main.tex
%% Authors: Petros Chanas, Konstantinos Vasilopoulos
%% Last update: April, 2022
%% Description: Provides our assignment 
%% regarding Open Source Software Quality Testing
%%and parameters of rating the performance and quality
%%of software developed of open source applications
%%%%%%%%%%%%%%%%%%%AUEB2021-2022%%%%%%%%%%%%%%%%%%%%%
%% Character encoding: UTF-8
%%%%%%%%%%%%%%%%%%%%%%%%%%%%%%%%%%%%%%%%%%%%%%%%%%%%%


\documentclass{article}
% Set the font (output) encoding
\usepackage[LGR]{fontenc}

% Greek-specific commands
\usepackage[greek]{babel}
\usepackage{graphicx}
\graphicspath{ {./images/} }
\title{\textbf{\textlatin{Open source software quality}}}
\author{{\textit{Κωνσταντίνος Βασιλόπουλος\(\bullet\)3180018}\\*\textit{Πέτρος Χάνας\(\bullet\)3170173}}\\*\includegraphics[width=4cm, height=3cm]{oSource}}
\date{\today}
\begin{document}
\maketitle

\begin{abstract}
Σε αυτήν την εργασία θα αναφερθούμε στους τρόπους ανάπυτηξης ελεύθερου λογισμικού
και στα ενδεχόμενα προβλήματα τα οποία προκύπτουν ως προς τον προγραμματισμό και το 
\textlatin{debugging} αυτού καθ'όλη τη διάρκεια ζωής του. Επίσης θα μελετήσουμε τους 
τρόπους αξιολόγησης του λογισμικού από τρίτους παράγοντες όπως εταιρείες, δημόσιοι οργανισμοί 
αλλά και από τους ίδιους τους χρήστες. Ουσιαστικα, η ποιότητα του ανοιχτού-ελεύθεροου λογισμικού 
συνίσταται από κάποιους βασικούς πυλώνες άμεσα εξαρτώμενος από το ίδιο το λογισμικό όπως η ταχύτητά
του, η αξιοπιστία του, η ευκολία στην χρήση του αλλά και την περιήγησή του και πολλά άλλα. Εν ολίγοις, 
θα αναλύσουμε όλες τις πτυχές ανάπτυξης και αξιολόγησης της ποιότητας λογισμικού ανοιχτού κώδικα 
διανθίζοντας όλα τα επιμέρους στοιχεία που συγκροτούν αλλά και καθιστούν το λογισμικό έτοιμο προς διάθεση 
στο ευρύ κοινό.

\end{abstract}

\section{Eισαγωγή}
Αυτό είναι το πρώτο τμήμα του εγγράφου. 
Είναι μια εισαγωγική παράγραφος.

\section{Τι είναι το λογισμικό ανοιχτού κώδικα;}
 
%%perigrafh open source code by definition
%%famous examples etc.

\section{Τρόποι Ανάπτυξης}
\subsection{Σύντομη Αναφορά στα \textlatin{Linux}}
%%linux milestone

\end{document}