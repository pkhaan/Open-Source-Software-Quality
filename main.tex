%%%%%%%%%%%%%%%%%%%%%%%%%%%%%%%%%%%%%%%%%%%%%%%%%%%%%
%% File: main.tex
%% Authors: Petros Chanas, Konstantinos Vasilopoulos
%% Last update: April, 2022
%% Description: Provides our assignment 
%% regarding Open Source Software Quality Testing
%%and parameters of rating the performance and quality
%%of software developed of open source applications
%%%%%%%%%%%%%%%%%%%AUEB2021-2022%%%%%%%%%%%%%%%%%%%%%
%% Character encoding: UTF-8
%%%%%%%%%%%%%%%%%%%%%%%%%%%%%%%%%%%%%%%%%%%%%%%%%%%%%


\documentclass{article}
% Set the font (output) encoding
\usepackage[LGR]{fontenc}

% Greek-specific commands
\usepackage[greek]{babel}
\usepackage{graphicx}
\usepackage[T1]{fontenc}
\usepackage{tgtermes}
\graphicspath{ {./images/} }
\title{\textbf{\textlatin{Open source software quality}}}
\author{{\textit{Κωνσταντίνος Βασιλόπουλος\(\bullet\)3180018}\\*\textit{Πέτρος Χάνας\(\bullet\)3170173}}\\*\includegraphics[width=4cm, height=3cm]{oSource}}
\date{\today}
\begin{document}
\maketitle

{\fontfamily{cmss}\selectfont
\begin{abstract}
Σε αυτήν την εργασία θα αναφερθούμε στους τρόπους ανάπυτηξης ελεύθερου λογισμικού
και στα ενδεχόμενα προβλήματα τα οποία προκύπτουν ως προς τον προγραμματισμό και το 
\textlatin{debugging} αυτού καθ'όλη τη διάρκεια ζωής του. Επίσης θα μελετήσουμε τους 
τρόπους αξιολόγησης του λογισμικού από τρίτους παράγοντες όπως εταιρείες, δημόσιοι οργανισμοί 
αλλά και από τους ίδιους τους χρήστες. Ουσιαστικα, η ποιότητα του ανοιχτού-ελεύθεροου λογισμικού 
συνίσταται από κάποιους βασικούς πυλώνες άμεσα εξαρτώμενος από το ίδιο το λογισμικό όπως η ταχύτητά
του, η αξιοπιστία του, η ευκολία στην χρήση του αλλά και την περιήγησή του και πολλά άλλα. Εν ολίγοις, 
θα αναλύσουμε όλες τις πτυχές ανάπτυξης και αξιολόγησης της ποιότητας λογισμικού ανοιχτού κώδικα 
διανθίζοντας όλα τα επιμέρους στοιχεία που συγκροτούν αλλά και καθιστούν το λογισμικό έτοιμο προς διάθεση 
στο ευρύ κοινό.
\end{abstract}

{\fontfamily{cmss}\selectfont\section{Τι είναι το λογισμικό ανοιχτού κώδικα\textlatin{;}}}
%%perigrafh open source code by definition
%%famous examples etc.
 Συχνά θα ακούσουμε ανθρώπους της πληροφορικής να αναφέρονται σε συγκεκριμένα προγράμματα ως «ανοιχτού κώδικα» ή «ελεύθερου λογισμικού». Εάν ένα πρόγραμμα είναι ανοιχτού κώδικα, ο πηγαίος κώδικας του είναι ελεύθερος διαθέσιμος στους χρήστες του. Οι χρήστες του - και οποιοσδήποτε άλλος - έχουν τη δυνατότητα να λάβουν αυτόν τον πηγαίο κώδικα, να τον τροποποιήσουν και να διανείμουν τις δικές τους εκδόσεις του προγράμματος. Οι χρήστες έχουν επίσης τη δυνατότητα να διανέμουν όσα αντίγραφα του αρχικού προγράμματος θέλουν. Οποιοσδήποτε μπορεί να χρησιμοποιήσει το πρόγραμμα για οποιονδήποτε σκοπό. δεν υπάρχουν χρεώσεις αδειοδότησης ή άλλοι περιορισμοί στο λογισμικό. Για παράδειγμα, το \textlatin{Ubuntu Linux} είναι ένα λειτουργικό σύστημα ανοιχτού κώδικα. Μπορείτε να κατεβάσετε το \textlatin{Ubuntu}, να δημιουργήσετε όσα αντίγραφα θέλετε και να τα δώσετε στους φίλους σας. Μπορείτε να εγκαταστήσετε το Ubuntu σε απεριόριστο αριθμό υπολογιστών. Μπορείτε να δημιουργήσετε αντίγραφα του δίσκου εγκατάστασης του \textlatin{Ubuntu} και να τα διανείμετε. Εάν είχατε ιδιαίτερα κίνητρα, θα μπορούσατε να κατεβάσετε τον πηγαίο κώδικα για ένα πρόγραμμα στο Ubuntu και να το τροποποιήσετε, δημιουργώντας τη δική σας προσαρμοσμένη έκδοση αυτού του προγράμματος - ή του ίδιου του \textlatin{Ubuntu}. Όλες οι άδειες ανοιχτού κώδικα σάς επιτρέπουν να το κάνετε αυτό, ενώ οι άδειες κλειστού κώδικα επιβάλλουν περιορισμούς σε εσάς.\\*
\\*
\section{Ιστορική Αναδρομή}
Κατά τη διάρκεια των πρώτων δεκαετιών της πληροφορικής, ήταν ο κανόνας και όχι η εξαίρεση η κοινή χρήση αναγνώσιμου πηγαίου κώδικα από τον άνθρωπο. Στις δεκαετίες του 1950 και του 1960 σχεδόν όλο το λογισμικό που παρήχθη από ακαδημαϊκούς και εταιρικά ερευνητικά εργαστήρια, όπως τα \textlatin{Bell Labs} της  \textlatin{AT&T}, εργάστηκαν σε συνεργασία. Όλοι είχαν μακροχρόνιες παραδόσεις ανοιχτού χαρακτήρα και συνεργασίας, άτυπες για τον ακαδημαϊκό χώρο, και ως εκ τούτου, ακόμη κι αν το λογισμικό δεν ήταν επίσημα διαθέσιμο στο κοινό, ο πηγαίος κώδικας τους ήταν ευρέως κοινός. Οι εταιρείες ηλεκτρονικών υπολογιστών διένειμαν επίσης τον πηγαίο κώδικα του λογισμικού που έστειλαν μαζί με το υλικό, για δύο λόγους. Πρώτον, δεν έβλεπαν το λογισμικό ως εμπόρευμα προς πώληση μέσω αδειοδότησης. Δεύτερον, οι χρήστες τροποποιούσαν συχνά οι ίδιοι το λογισμικό επειδή δεν θα λειτουργούσε σε διαφορετικό υλικό ή λειτουργικό σύστημα χωρίς κάποια παρέμβαση στον "κορμό" του, καθώς και για να διορθώσουν σφάλματα ή να προσθέσουν νέες λειτουργίες.

Το να έχουν διαθέσιμο τον πηγαίο κώδικα ήταν μια αναγκαιότητα για τους κατασκευαστές υπολογιστών, καθώς η δημιουργία διαφορετικών \textlatin{binary files} (ή μεταγλωττισμένου κώδικα) για διαφορετικό υλικό δεν ήταν καθόλου πρακτική και 
συχνά δημιουργούσε μεγάλες επιπλοκές. Ορισμένα πανεπιστήμια είχαν ακόμη και μια πολιτική που απαιτούσε όλα τα λογισμικά που είναι εγκατεστημένα στους υπολογιστές στα εργαστήριά τους να συνοδεύονται από δημοσιευμένους πηγαίους κώδικες.

Το 1953, το τμήμα \textlatin{UNIVAC} του \textlatin{Remington Rand} ανέπτυξε την πρώτη παρουσία ελεύθερου λογισμικού ανοιχτού κώδικα που ονομάζεται  \textlatin{A-2 (Arithmetic Language v2 system)}, το οποίο κυκλοφόρησε στους πελάτες τους μαζί με τον πηγαίο κώδικα. Προσκλήθηκαν επίσης να στείλουν πίσω τις βελτιώσεις τους. Αργότερα, το πρώτο λειτουργικό σύστημα της  \textlatin{IBM}, ο κώδικας του  \textlatin{IBM 704} διανεμήθηκε με όλους τους μεγάλους υπολογιστές τους. Οργανισμοί όπως η \textlatin{IBM}, η \textlatin{DEC} και η \textlatin{General Motors} δημιουργούν ομάδες χρηστών για να διευκολύνουν την κοινή χρήση κώδικα μεταξύ των χρηστών, ακαδημαϊκών και άλλων
παραγόντων του κλάδου. 

Με την πάροδο των δεκαετιών και την συνεχώς αναπτυσσόμενη βιομηχανία της τεχνολογίας ποικίλοι παράγοντες απαιτούσαν την διανομή, σε μεγαλύτερο βεληνεκές, ανοιχτού κώδικα λογισμικού για την κάλυψη των ήδη υπαρχουσών αναγκών. O ανοιχτός κώδικας απεδείχθη ως 
το κύριο πλεονέκτημα μικρών ομάδων προγραμματιστών για την ανάπτυξη μεγάλων και "ευρυζωνικών", τρόπον τινά, εφαρμογών και λογισμικού. Χαρακτηριστικότερο ίσως παράδειγμα αυτής της αυτοδυναμίας ανάπτυξης αποτελεί ο \textlatin{Linus Torvalds} που 
στις αρχές της δεκαετίας του 1990 κατάφερε να προγραμματίσει έναν καινούριο \textlatin{kernel} βασισμένο εκ των προτέρων αλλά όχι εξ'ολοκλήρου στο λειτουργικό 
\textlatin{UNIX}. Έτσι, η κοινότητα του ελεύθερου λογισμικού έλαβε το πρώτο πλήρες δωρεάν λειτουργικό σύστημα με τον πυρήνα του \textlatin{Linus Torvalds} σε συνδυασμό με το λειτουργικό σύστημα \textlatin{GNU}. Το \textlatin{Debian}, που ιδρύθηκε από τον \textlatin{Ian Murdock} το 1993, δεσμεύτηκε στις αρχές \textlatin{GNU} και \textlatin{FSF} του ελεύθερου λογισμικού. Μεταγενέστερα παραδείγματα ελεύθερου λογισμικού αποτελούν οι διανομές ή \textlatin{distros} του \textlatin{Linux} και πληθώρα πακέτων και προγραμμάτων διαθέσιμα ενδολειτουργικά όπως π.χ. το \textlatin{inkcape}, το \textlatin{gcalculator}, τα \textlatin{KDE} πακέτα και πολλά άλλα.

\subsection{Τι είναι το  \textlatin{Linux}}
%%Description of OSI as the future of Systems Intercommunications.
%%Το μοντέλο διασύνδεσης ανοιχτών συστημάτων \textlatin{OSI} είναι ένα εννοιολογικό μοντέλο που δημιουργήθηκε από τον Διεθνή Οργανισμό Τυποποίησης που επιτρέπει σε διάφορα συστήματα επικοινωνίας να επικοινωνούν χρησιμοποιώντας τυπικά πρωτόκολλα. Σε απλά αγγλικά, το \textlatin{OSI}παρέχει ένα πρότυπο για διαφορετικά συστήματα υπολογιστών ώστε να μπορούν να επικοινωνούν μεταξύ τους.

%%Το μοντέλο \textlatin{OSI} μπορεί να θεωρηθεί ως μια καθολική γλώσσα για τη δικτύωση υπολογιστών. Βασίζεται στην ιδέα του διαχωρισμού ενός συστήματος επικοινωνίας σε επτά αφηρημένα επίπεδα, το καθένα στοιβαγμένο στο τελευταίο.

Το \textlatin{Linux} είναι μια από τις δημοφιλείς εκδόσεις του λειτουργικού συστήματος \textlatin{UNIX}. Είναι ανοιχτού κώδικα καθώς ο πηγαίος κώδικας του είναι δωρεάν διαθέσιμος. Το \textlatin{Linux} σχεδιάστηκε λαμβάνοντας υπόψη τη συμβατότητα \textlatin{UNIX}. Η λίστα λειτουργιών του είναι αρκετά παρόμοια με αυτή του \textlatin{UNIX}. Ακριβώς όπως τα \textlatin{Windows}, \textlatin{iOS} και \textlatin{Mac OS}, το \textlatin{Linux} είναι μια από τις πιο δημοφιλείς πλατφόρμες στον πλανήτη. Το \textlatin{Android}, τροφοδοτείται από το λειτουργικό σύστημα \textlatin{Linux} και σχεδόν καθ'ολολκληρίαν όλα τα λειτουργικά υπεύθυνα για την διαχείριση και την λειτουργία \textlatin{servers} δουλεύουν πάνω σε \textlatin{Linux}. 

Η υιοθέτηση του Linux στην μαζική παραγωγή, αντί της πρότερης χρήσης αποκλειστικά και μόνο από ερασιτέχνες, άρχισε να απογειώνεται για πρώτη φορά στα μέσα της δεκαετίας του 1990 στην κοινότητα των υπερυπολογιστών, όπου οργανισμοί όπως η \textlatin{NASA} άρχισαν να αντικαθιστούν τις ολοένα και πιο ακριβές μηχανές τους με ομάδες φθηνών εμπορευματικών υπολογιστών με \textlatin{Linux}. Η εμπορική χρήση ξεκίνησε όταν η \textlatin{Dell} και η \textlatin{IBM}, ακολουθούμενη από τη \textlatin{Hewlett-Packard}, άρχισαν να προσφέρουν υποστήριξη \textlatin{Linux} για να ξεφύγουν από το μονοπώλιο της \textlatin{Microsoft} στην αγορά λειτουργικών συστημάτων επιτραπέζιων υπολογιστών.

Σήμερα, τα συστήματα \textlatin{Linux} χρησιμοποιούνται σε όλους τους υπολογιστές, από τα ενσωματωμένα συστήματα έως σχεδόν όλους τους υπερυπολογιστές, και έχουν εξασφαλίσει μια θέση σε εγκαταστάσεις διακομιστή όπως η δημοφιλής στοίβα εφαρμογών \textlatin{LAMP}. Η χρήση των διανομών \textlatin{Linux} σε οικιακούς και εταιρικούς επιτραπέζιους υπολογιστές έχει αυξηθεί. Οι διανομές \textlatin{Linux} έχουν γίνει επίσης δημοφιλείς στην αγορά netbook, με πολλές συσκευές να αποστέλλονται με προσαρμοσμένες διανομές \textlatin{Linux} εγκατεστημένες και η \textlatin{Google} να κυκλοφορεί το δικό της \textlatin{Chrome OS} σχεδιασμένο για \textlatin{netbook}.


\subsubsection{\textlatin{Ubuntu}}

\subsubsection{\textlatin{Debian}}

\subsection{\textlatin{Mozilla Firefox}}



\section{Ποιότητα Λογισμικού}
\subsection{Ορισμός}
Η ποιότητα λογισμικού είναι μια αφηρημένη έννοια που γίνεται αντιληπτή και ερμηνευόμενη διαφορετικά
με βάση τις προσωπικές απόψεις και τα ενδιαφέροντά του. Για να λυθεί αυτή η ασάφεια, το \textlatin{ISO/IEC9126} (Διεθνής Οργανισμός Τυποποίησης 2001) παρέχει ένα πλαίσιο για
την αξιολόγηση της ποιότητας του λογισμικού. Ορίζει έξι χαρακτηριστικά ποιότητας λογισμικού, συχνά
αναφέρονται ως ποιοτικά χαρακτηριστικά:
\begin{itemize}
\item \textbf{Λειτουργικότητα}: Εάν το λογισμικό εκτελεί τις απαραίτητες λειτουργίες.
\item \textbf{Αξιοπιστία}:
\item \textbf{Χρηστικότητα}:
\item \textbf{Αποτελεσματικό}:
\item \textbf{Συντηρησιμότητα}:
\item \textbf{Φορητότητα}:
\end{itemize}

\subsection{Tρόποι Ελέγχου Ποιότητας}%petros
%ιστορικη αναδρομη-αναφορά στον ελεγχο ποιότητας
\subsection{Εργαλεία Ελέγχου Ποιότητας}

\subsection{Ταχύτητα}

% Costas
\subsection{Μοντέρνος Έλεγχος Ποιότητας}



\subsubsection{Λογισμικό Ελέγχου Πηγαίου Κώδικα}

\subsubsection{Αποφυγή Κακόβουλου Λογισμικού}

\textbf{\section{Αναμενόμενα Αποτελέσματα}}







\section{Επίλογος}






\section{Βιβλιογραφία}

} %font
\end{document}